Wykres \ref{fig:optimum} uświadamia nam już na etapie testowania algorytmu pakowania do pojemników, że zastosowanie relaksacji daje lepsze wyniki niż optymalne bez zastosowania relaksacji. Ciekawy wniosek można wysunąć obserwując wyniki algorytmu BinPack bez relaksacji, gdyż aż w ponad 80\% przypadków testowych wyznaczył optimum, a przypadki w których mu się to nie udało, nie odbiegały znacząco od rozwiązania optymalnego.

Analizując wykres porównujący czas szukania optimum z algorytmem BinPack \ref{fig:effi}, wywnioskować można, że znajdywanie rozwiązania optymalnego było dużo bardziej czasochłonne. Policzenie najdłuższych przypadków zajmowało 2-3 godziny, a biorąc pod uwagę ograniczony rozmiar zamówień można się spodziewać że przypadki produkcyjne (tysiące elementów zamawianych) zajmowałyby dni. Za to czas dla algorytmów opartych o pakowanie mieścił się w dziesiątkach sekund.

Najbardziej interesujące są jednak wyniki porównujące działanie algorytmów BinPack i metody generacji kolumn. Na każdym z nich zaobserwować można znacząca przewagę programowania liniowego. We wszystkich przypadkach udało mu się znaleźć lepsze rozwiązania do tego w czasie nieporównywalnie szybszym. Zmiana rozmiaru zamówienia praktycznie nie wpływała w ogóle na czas wykonywania. Zachowanie programu AMPL zostało sprawdzone nawet na zamówieniu o rozmiarze 10000000. Rozwiązane zostało w czasie 2 sekund. Spowodowane jest to tym, że metoda generacji kolumn swój czas poświęca tylko na odnalezienie odpowiedniego zestawu wzorców. W procesie tym nie ma znaczenia ilość zamawianych prętów, a jedynie ich różnorodność w zamówieniu. 