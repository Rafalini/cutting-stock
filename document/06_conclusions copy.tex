W ramach pracy zaimplementowano 2 różne algorytmy rozwiązujące problem cięcia stali z możliwością relaksacji prętów. Pierwszy z nich rozwiązywał zadanie jako problem programowania liniowego korzystając z metody generacji kolumn. Drugi zaś stosował algorytm pakowania plecakowego, korzystając z heurystyki First Fit Decreasing. Zadowalające wyniki zostały uzyskane w przypadku obu tych rozwiązań, jednakże to stosujące metodę generacji kolumn zdecydowanie okazało się lepsze w rozwiązywaniu tego typu problemu. Przedstawiono również zalety korzystania z relaksacji, gdy jest to możliwe. Stosowanie jej dawało, jak się można było spodziewać, wyniki lepsze, nie powodując przy wydłużenia czasu wykonywania programu.