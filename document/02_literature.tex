Praca autorstwa Gilmore'a i Gomory'ego stanowi \cite{linear-programming-gilmore} kamień milowy w rozwiązaniu problemu cięcia jednowymiarowego. Opracowana metoda oparta na programowaniu liniowym pozwoliła na minimalizację strat materiałowych poprzez optymalne wykorzystanie dostępnych surowców. W artykule rozważany jest przypadek, w którym na wejściu dostępny jest zestaw różnej długości materiału z którego dalej wycinane są elementy o pożądanym wymiarze. W ogólności problem jest dość złożony ze względu na liczbę zmiennych decyzyjnych, która rośnie wykładniczo. W ogólności problem ten uniemożliwia wyznaczanie optimum globalnego dla bardzo złożonych przypadków tą metodą. Autorzy wskazują kilka sposobów na redukcję liczby zmiennych decyzyjnych takich jak: tymczasowe odrzucenie ograniczenia, rozwiązanie zadania w dziedzinie całkowitoliczbowej, a następnie poszukiwanie rozwiązania w sąsiedztwie rozwiązania niecałkowitoliczbowego. Kolejnym pomysłem jest modyfikacja zapisu problemu w postaci macierzowej, tak aby usprawnić działanie algorytmu SIMPLEX - czyli metoda generacji kolumn. Polega ona na stopniowym rozwiązywaniu problemu z coraz to większą liczbą zmiennych (ponieważ zapisanie całego zadania wymagałoby bardzo dużej ich ilości). Chodzi o iteracyjne rozwiązywanie coraz większego problemu. Takie rozwiązanie eliminuje konieczność pełnego przeszukiwania przestrzeni rozwiązań dzięki czemu charakteryzuje się wysoką efektywnością czasową. 
Niemniej jednak, istnieje istotna wada tego podejścia, wynikająca z operowania na liczbach całkowitoliczbowych, co implikuje, że wynik końcowy, będący liczbą rzeczywistą, wymaga przybliżenia. Dodatkowym ograniczeniem tego modelu jest nieodzowna konieczność wyrażania długości w jednostkach całkowitych. W związku z tym, aby uwzględnić pręty o niecałkowitych długościach, takie jak 7,5 m, konieczne jest przeskalowanie wszystkich długości do jednostek bardziej precyzyjnych, takich jak milimetry czy centymetry. Z punktu widzenia niniejszej pracy, skalowanie nie stanowi problemu, dlatego opracowana metoda stanowi solidny punkt wyjścia, pozwalając na potencjalne rozwinięcie go poprzez dodanie dodatkowych ograniczeń dotyczących relaksacji, co może przynieść pożądane efekty, zgodne z celem badawczym tej pracy.

Bardzo podobny problem porusza praca \cite{optimization-cloth}, w której autorzy również rozważają model programowania liniowego, całkowitoliczbowego do minimalizowania odpadków, w przypadku w którym wejściowe elementy bazowe są różnej długości. Artykuł opisuje implementacje powyższych metod wprowadzonych w pracy \cite{linear-programming-gilmore} i jego efektem jest implementacja w FORTRANie.

Kolejnym godnym uwagi badaniem, które analizuje aspekty problematyki cięcia materiałów, jest praca poświęcona minimalizacji wzorców cięcia \cite{patterns-reduction}. Autorzy tego opracowania identyfikują istotny fakt, że dla każdego zamówienia możliwe jest wygenerowanie różnorodnych wzorców cięcia, co oznacza różne kombinacje długości prętów bazowych potrzebnych do wycięcia zamówionych elementów. Teoretycznie istnieje ogromna liczba potencjalnych kombinacji, jednakże w praktyce wiele z tych kombinacji jest nieefektywnych pod kątem minimalizacji marnotrawstwa materiału. Zatem autorzy dekomponują problem na dwa podproblemy, pierwszy to wygenerowanie wzorców cięcia, kolejny to określenia liczby poszczególnych wzorców w sposób minimalizujący zadanie pierwotne. Następnie autorzy podchodzą iteracyjne do zagadnienia znalezienia globalnego optimum poprzez stopniowe zwiększanie liczby wzorców. Czyli dla T=1,2,3...n, generują T wzorców, rozwiązują problem optymalnego cięcia dla tego zestawu wzorców i porównują je z najlepszym osiągniętym dotychczas wynikiem, a proces ten powtarzają dopóki właśnie obliczane rozwiązanie jest lepsze od poprzednich. Atutem tego podejścia jest łatwość uwzględnienia kosztu ustawiania maszyn do cięcia, który może zmodyfikować wyjściową funkcję kosztu, ale który dość dobrze odwzorowuje rzeczywisty stan rzeczywistości - w którym może się okazać że optymalne rozwiązanie wykorzystuje zbyt wiele różnych cięć i staje się ciężkie do obsłużenia bo wymaga zmieniania nastaw maszyn do cięcia. Autorzy stosują model statystyczny, który generuje najbardziej prawdopodobne wzorce na podstawie zamówienia, niemniej jednak zagadnienie doboru wzorców stanowi wyzwanie i jest polem do usprawniania całego rozwiązania. Według autorów możliwe jest zastosowanie modelu sztucznej sieci neuronowej, który ma na celu przewidywanie lub, dokładniej, klasyfikację wzorców cięcia na podstawie ich efektywności i przydatności jeszcze bez rozwiązywania samego problemu cięcia dla danego zestawu wzorców. 

Do problemu minimalizacji ilości zamówionych prętów bazowych nieco inaczej podchodzą autorzy pracy \cite{structural-tubes}, w której skupiają się na takim wycinaniu elementów, aby odpady nadawały się do dalszego przetworzenia. Ponownie problem formułują jako zagadnienie programowania liniowego całkowitoliczbowego, ale nie minimalizują pojedynczych zamówień, tylko minimalizują ilość niemożliwych do użytku odpadów. To oznacza że może opłacać się zamówić większą ilość prętów bazowych, żeby zastosować takie cięcia które pozwolą na uzyskanie minimalnego odpadu nie możliwego do użycia, a nadmiar fragmentów które nie są potrzebne w tym zamówieniu zostanie wykorzystany do innego. Zatem tutaj autorzy minimalizują ilość odpadów z punktu widzenia przedsiębiorstwa, które nieustannie realizuje jakieś zamówienia i ma możliwość niejako poprzesuwania części z jednego zamówienia do drugiego i ma gwarancję wykorzystania fragmentów, które zostają z cięcia jeśli są większe niż jakaś dana wielkość.


Zupełnie odmienne podejście do rozwiązania problemu cięcia zastosowali autorzy artykułu \cite{bin-packing}, w którym przedstawili problem wycinania jako problem plecakowy, knapsack problem, a właściwie jego odmianę, w której pakowania dokonujemy wiele razy - bin pack problem. Zadanie sprowadza się do rozplanowania pakowania pożądanych długości materiału, do jak najmniejszej ilości długości bazowych materiału. Pojemnikami stają się tutaj bazowe długości prętów, a elementami które umieszczamy w pojemnikach, poszczególne długości które chcemy uzyskać. Jednak samodzielnie jest to problem NP-trudny, zatem nie ma bezpośredniego rozwiązania, istnieją tylko przybliżenia tego problemu osiągające optimum lokalne. Problem bliźniaczy, decyzyjny, odpowiadający na pytanie czy dany zestaw elementów do spakowania można umieścić w zadanej ilości pojemników o pewnym rozmiarze, jest NP-zupełny.

Bardzo pomocną pracą przy testowaniu naszych rozwiązań okazała się \cite{problem-generator}, w której autorzy proponują algorytm  do generowania przypadków testowych, na których uruchamialiśmy testy naszych algorytmów.
