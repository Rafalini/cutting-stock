The work by Gilmore and Gomory represents a \cite{linear-programming-gilmore} milestone in solving the one-dimensional slicing problem. The developed method based on linear programming allowed for minimizing material losses through optimal use of available raw materials. The article considers the case in which a set of different lengths of material is available at the input, from which elements of the desired size are further cut. In general, the problem is quite complex due to the number of decision variables, which increases exponentially. In general, this problem makes it impossible to determine the global optimum for very complex cases using this method. The authors indicate several ways to reduce the number of decision variables, such as: temporarily rejecting the constraint, solving the problem in the integer domain, and then searching for a solution in the vicinity of a non-integer solution. Another idea is to modify the problem in matrix form to improve the operation of the SIMPLEX algorithm - i.e. the column generation method. It involves gradually solving a problem with an increasing number of variables (because writing down the entire task would require a very large number of them). It's about iteratively solving an increasingly larger problem. This solution eliminates the need to fully search the solution space, making it highly time-efficient.
However, there is a significant drawback to this approach, resulting from the operation on integer numbers, which implies that the final result, which is a real number, requires approximation. An additional limitation of this model is the inherent need to express length in integer units. Therefore, to account for bars with non-integer lengths, such as 7.5 m, it is necessary to scale all lengths to more precise units, such as millimeters or centimeters. From the point of view of this work, scaling is not a problem, therefore the developed method is a solid starting point, allowing it to be potentially expanded by adding additional relaxation constraints, which may bring the desired effects, consistent with the research goal of this work.

A very similar problem is addressed in the work \cite{optimization-cloth}, in which the authors also consider an integer linear programming model for minimizing waste, in the case where the input base elements are of different lengths. The article describes the implementation of the above methods introduced in the work \cite{linear-programming-gilmore} and its effect is the implementation in FORTRAN.

Another noteworthy study that analyzes aspects of material cutting is the work on minimizing cutting patterns \cite{patterns-reduction}. The authors of this study identify the important fact that for each order it is possible to generate various cutting patterns, which means different combinations of the lengths of the base bars needed to cut the ordered elements. Theoretically there are a huge number of potential combinations, however in practice many of these combinations are ineffective in minimizing material waste. Therefore, the authors decompose the problem into two sub-problems, the first one is to generate cutting patterns, the next one is to determine the number of individual patterns in a way that minimizes the original task. The authors then iteratively approach the problem of finding the global optimum by gradually increasing the number of patterns. That is, for T=1,2,3...n, they generate T patterns, solve the problem of optimal cutting for this set of patterns and compare them with the best result achieved so far, and repeat this process until the calculated solution is better than the previous ones. The advantage of this approach is the ease of taking into account the cost of setting up cutting machines, which may modify the initial cost function, but which reflects the actual state of reality quite well - in which it may turn out that the optimal solution uses too many different cuts and becomes difficult to handle because it requires changing cutting machine settings. The authors use a statistical model that generates the most probable patterns based on the order, however, the issue of pattern selection is a challenge and is a field for improving the entire solution. According to the authors, it is possible to use an artificial neural network model that aims to predict or, more precisely, classify cutting patterns based on their effectiveness and suitability without solving the cutting problem itself for a given set of patterns.

The authors of \cite{structural-tubes} approach the problem of minimizing the number of ordered base bars in a slightly different way, in which they focus on cutting out elements so that the waste is suitable for further processing. They formulate the problem again as an integer linear programming problem, but they do not minimize individual orders, but rather minimize the amount of unusable waste. This means that it may be profitable to order a larger number of base bars to use cuts that will allow for minimal unusable waste, and the excess fragments that are not needed in this order will be used for another. Therefore, here the authors minimize the amount of waste from the point of view of a company that constantly carries out orders and has the ability to move parts from one order to another and is guaranteed to use the fragments that remain from cutting if they are larger than a given size.


A completely different approach to solving the cutting problem was used by the authors of the article \cite{bin-packing}, in which they presented the cutting problem as a knapsack problem, knapsack problem, or rather its variant in which packing is performed many times - bin pack problem. The task comes down to planning the packing of the desired material lengths to the smallest possible number of basic material lengths. Here, the basic lengths of the bars become the containers, and the elements that we place in the containers become the individual lengths that we want to obtain. However, it is an NP-hard problem on its own, so there is no direct solution, there are only approximations to this problem that reach a local optimum. The twin decision problem, answering the question whether a given set of items to be packed can be placed in a given number of containers of a certain size, is NP-complete.

\cite{problem-generator} turned out to be a very helpful work in testing our solutions, in which the authors propose an algorithm for generating test cases on which we ran tests of our algorithms.