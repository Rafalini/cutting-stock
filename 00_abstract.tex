Rozwiązanie problemu cięcia materiału (ang. textit{cutting stock problem}) - dalej przyjmujemy że tutaj konkretnie są cięte stalowe pręty - jest wyczerpująco omówione w literaturze, jednak w ramach tej pracy zajmujemy się jego zmodyfikowaną wersją. W porównaniu do oryginalnego problemu, zasadniczą różnicą jest to, że niektóre długości prętów można zmieniać w czasie obliczeń, jeśli umożliwia to poprawę wyniku rozwiązania. W tej pracy porównujemy dwa autorskie rozwiązania stosujące relaksację prętów, opierające się na algorytmie pakowania plecakowego oraz na metodzie generacji kolumn.